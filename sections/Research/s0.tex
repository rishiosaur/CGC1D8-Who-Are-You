\documentclass[../../main.tex]{subfiles}
\begin{document}
\section{General Information}
\begin{itemize}
    \item This photo was shot by \href{https://www.nationalgeographic.com/photography/proof/2017/02/philippine-drug-war/}{National Geographic}
    \item The story is about the unlawful punishments drug lords give to their members in the Phillipines, and the effects on the community.
    \item The story begins with a father named Rick Medina, and follows his path in finding out that his son, Ericardo, has been killed.
    \item That night, there were 7 more killings, and the day after, Ericardo's sister, Jhoy, went to the morgue to see all of them.
    \item There, she discovered something horrifying: the way they were killed.
    \begin{itemize}
        \item Upon further investigation, it was found that they all were killed the same way
        \item The murderers wrapped packing tape around their heads, then stabbed them with ice picks to pierce their lungs
    \end{itemize}
    \item These are just some of thousands of killings in the Phillipines, demonstrating the extent to which the problem lies.
    \item However, this problem does not just lie with drug users; people who've never even used drugs, but have aided police, can be killed.
    \begin{itemize}
        \item The photo depicts a bloody scene
        \begin{itemize}
            \item There is a figure with blood draining out of him in the bottom left hand corner
            \item In the background, there looks to be ordinary citizens
            \item On the right-hand side, there is a police photographer
        \end{itemize}
        \item The person bleeding out of his head was 41-year-old Angelito Luciano
        \item He was an assist to police in the area, seeking out drug lords and users
    \end{itemize}
\end{itemize}
\section{Coping}
\begin{itemize}
    \item The people of the Phillipines use many different methods to cope with the two main problems:
    \begin{itemize}
        \item Their loved one has been killed
        \item Their loved one was a user of drugs
        \begin{itemize}
            \item In the Phillipines, using drugs is highly looked down upon, despite the high drug rate.
        \end{itemize}
    \end{itemize}
    \item Death rituals have been a large part of Phillipine culture
    \item 
\end{itemize}
\section{Meaning of the photo}
\begin{itemize}
    \item For a very long time, the world has been looking to eradicate drugs
    \item This all started in the 90s, with Richard Nixon declaring that "public drug abuse is enemy number one"
    \item This led to a global campaign that seeked to eradicate drugs by incarcerating those that used them
    \item However, this ignores a very simple marketing foundation: supply and demand
    \begin{itemize}
        \item Unlike other industries, drug users cannot just stop using drugs.
        \item Once somebody gets addicted, it is \textit{very} hard to stop.
        \item As a result, when supply drops, demand does not go down with it
        \item This leads to very strong competitiveness in the market, and people will end up doing anything to get that product
    \end{itemize}
    \item The 'doing anything' part of that sentence does not just mean for consumers; as is evident in this photo, drug producers iwll also do anything to anybody that gets in their way
    \item It is scenarios like this that the world starts to realize the err of their ways
    \begin{itemize}
        \item The so-called "War on Drugs" has been claimed to be a complete waste of time, money, and human resources.
        \item Countries like Switzerland have stopped all forms of combatting the 'War on Drugs', and they have shown that human-to-human interaction and rehabilitation is the best way to combat drug use.
    \end{itemize}
    \item As more people start to figure out that parts of our world are inherently wrong, so our world will progress.
\end{itemize}

\end{document}